\documentclass[10pt]{article}

\usepackage{graphicx}
\usepackage{url}

\begin{document}

\title{Ibis Users Guide}

\author{The Ibis Group}

\maketitle

\section{Introduction}

This manual describes the steps required to run an application that uses
 the Ibis communication library. How to create such an application
is described in the IPL Programmers manual.

A central concept in Ibis is the \emph{Pool}. A Pool consists of one or
more Ibis instances, usually running on different machines. Each pool is
generally made up of Ibisses running a single distributed applications.
Ibisses in a pool can communicate with each other, and, using the
registry mechanism present in Ibis, can search for other Ibisses in the
same pool, get notified of Ibisses joining the pool, etc. To
coordinate Ibis pools a so-called \emph{Ibis Server} is used.

\section{The Ibis Server}

The Ibis Server is the Swiss-army-knife server of the Ibis project.
Services can be dynamically added to the server. By default, the Ibis
communication library comes with a registry service. This registry
service keeps track of pools, and can track multiple pools at the same
time.  The server also allows Ibisses to route traffic over the server
if no direct connection is possible between two instances due to
firewalls or NAT boxes.

The Ibis server is started with the \texttt{ibis-server} script which is
located in the Ibis \texttt{bin} directory.  Before starting an Ibis
application, an Ibis server needs to be running on a machine that is
accessible from all nodes participating in the Ibis run. The server
listens to a TCP port. The port number can be specified using the
\texttt{--port} command line option to the \texttt{ibis-server} script.
For a complete list of all options, use the \texttt{--help} option of
the script. One usefull options is the  \texttt{--event} option, which
makes the registry print out events (such as ibisses joining a pool).

\section{Running an Ibis Application}

When the Ibis Server is running, the Ibis application can now be
started. There are a number of requirements if Ibis is to function
correctly. We will discuss these in detail below.

\subsection{Add ipl.jar to the classpath}

An application interfaces to Ibis using the Ibis Portability Layer, or
\emph{IPL}. The code for this package is provided in a single jar file:
ipl.jar (appended with the version of ibis, for instance \texttt{ipl-2.0.jar}.
This jar file needs to be added to the classpath of the application.

\subsection{Provide the Ibis implementations}

The IPL loads the Ibis implementation used to actually communicate
dynamically. These implementations (and their dependancies) can be
provided in two alternative ways:

\begin{enumerate} 
\item add the jar files of the implementations and their dependancies to
      the classpath 
\item set the \texttt{ibis.implementation.path} system property to the
      location of the Ibis implementations and dependancies.  
\end{enumerate}

System properties can be set in Java using the \texttt{-D} option of the
\texttt{java} command. The ibis.implementation.path property is a list
of directories, seperated by the default path seperator of your
operating system. In Unix, this is the \emph{;} character, in Windows it
is a \emph{:}.

\subsection{Configure Log4j}

Ibis uses the Log4J library of the Apache project to print debugging
information, warnings, and error messages. This library must be
initialized. A configuration file can be specified using the
\texttt{log4j.configuration} system property. For example, to use a file
named \texttt{log4j.properties} in the current directory, set
\texttt{log4j.configuration} to \texttt{file:log4j.properties}

\subsection{Set the location of the Server}

To communicate with the registry service, the address of the Ibis server
is needed. It must be specified using the \texttt{ibis.server.address}
system property. The full address needed is printed on startup of the Ibis
server. 

For convenience, it is also possible to only provide a adress,
port number pair, e.g. \texttt{machine.domain.com:5435} or even simply a
host, e.g. \texttt{localhost}. In this case, the port number is implied
(the default is 8888). The port number provided must match the one given
to ibis-server on startup using the \texttt{--port} option.

The Ibis Server is a single point which needs to be reachable from every
Ibis instance. Since sometimes this is not possible due to firewalls,
additional \emph{hubs} can be started, creating a routing infrastructure
for the Ibis instances. These hubs can be started by using ibis-server
script with the \texttt{--hub-only} option. See the \texttt{--help}
option of the ibis-server script for more information. 

When additional hubs are started, these must also be given to the ibis
instances. This can be done using the using the
\texttt{ibis.hub.addresses} property. Ibis expects a comma seperated
list of addresses of hubs.

\subsection{Set the properties of the pool}

Each pool in Ibis has a unique name. A server can service multiple
concurrent pools, and this name is used to determine which ibis
instances belong to which pool at the server. The pool name can be set
using the \texttt{ibis.pool.name} system property.

Sometimes, pools have a fixed size. In these so-called \emph{closed
world} pools, the number of ibises in the pool is also needed for ibis
to function correctly. The size must be set using the
\texttt{ibis.pool.size} property. This property is normally not needed,
and Ibis will print an error when it requires this property.

\section{The ibis-run script}

To make it slightly easier to run a Ibis application, a
\texttt{ibis-run} script is provided with the ibis distribution. The
\texttt{ibis-run} script can be used as follows

\begin{center}
\texttt{ibis-run} \emph{java-flags class params}
\end{center}

This script performs the first three steps needed to run an application
using Ibis. It adds the ipl.jar and all Ibis implementations to the
classpath, and configures log4j. It then runs \texttt{java} with any
command line options given to it. So, any additional options for Java,
the main class and any application parameters can be provided as if
\texttt{java} was called directly.

The \texttt{ibis-run} script needs the location of the Ibis
distrobution. This must be provided using the IBIS\_HOME environment
variable.

\section{Example}

To illustrate running an Ibis application we will use a simple Hello
world application. This application is started twice on a single
machine. One instance will send a small message to the other, which will
print it.

\subsection{Compiling the example}

Some example applications for the Ibis communication library are
provided with the Ibis distribution, in the \texttt{examples} directory.
They can be compiled using \texttt{ant}, the make-like system for Java.
Running \texttt{ant} in the examples directory should compile these
applications. 

Alternatively, they can be compiled using only javac. The sources are
located in the \texttt{src} directory of the examples. Be sure to add
\texttt{ipl.jar} from the \texttt{lib} directory of the distribution to
the classpath.

\subsection{Running the example}

We will now run the example. All code below assumes the IBIS\_HOME
environment variable is set to the location of the Ibis distribution.

First, we need a ibis-server. Start a shell and
run the \texttt{ibis-server} script:
\noindent
{\small
\begin{verbatim}
$ $IBIS_HOME/bin/ibis-server --events
\end{verbatim}
}
\noindent

As you can see, we provided the \texttt{--events} option to get some
extra information on when ibises join and leave the pool.

Next, we will start the application twice. One instance will act as the
"server", and one the "client". The application will determine who is
the server and who is the client automatically. Therefore we can simply
start the application using the same command line. In two different
shells type:

\noindent
{\small
\begin{verbatim}
$ CLASSPATH=$IBIS_HOME/examples/lib/ipl-examples-2.0.jar $IBIS_HOME/bin/ibis-run \
    -Dibis.server.address=localhost -Dibis.pool.name=test \
    ibis.ipl.examples.Hello
\end{verbatim}
}
\noindent

This sets the CLASSPATH environment variable to the jar file of the
application, and called ibis-run. Now, you should have the two running
instances of your application. One of them should print:

\noindent {\small \begin{verbatim} Server received: Hi there
\end{verbatim} } \noindent 

As said, the ibis-run script is mearly provided for convenience. To run
the application without ibis-run, the following command can be used:

\noindent
{\small
\begin{verbatim}
$ java \
    -cp
    $IBIS_HOME/lib/ipl-2.0.jar:$IBIS_HOME/examples/lib/ipl-examples-2.0.jar \
    -Dibis.impl.path=$IBIS_HOME/lib \
    -Dibis.server.address=localhost \
    -Dibis.pool.name=text \
    -Dlog4j.configuration=file:$IBIS_HOME/log4j.properties \
    ibis.ipl.examples.hello.Hello
\end{verbatim}
}
\noindent

In this case, we use the \texttt{ibis.impl.path} property to supply Ibis
with the jar files of the Ibis implementations. Alternatively, they
could also all be added to the classpath.

\section{Further Reading}

The Ibis web page \url{http://www.cs.vu.nl/ibis} lists all
the documentation and software available for Ibis, including papers, and
slides of presentations.

For detailed information on developing an Ibis application see the
Programmers Manual, available in the docs directory of the Ibis
distribution

\end{document}
